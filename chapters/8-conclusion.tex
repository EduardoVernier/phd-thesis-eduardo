\chapter{Conclusion}
\label{ch:conclusion}


% -we examined the evaluation and improvement of viz algos for abstract dynamic data.
% -we considered 2 main instances of this data: hierarchies and high-dim datasets.
% -in both cases, we found out that quality can be described by two main criteria: vis quality and stability
% -in both cases, for vis quality, we could reuse existing metrics from vizzes for static data of those types
% -in both cases, for stability, there was nothing; we could propose new metrics for that, which are interestingly almost the SAME for the 2 kinds of data; they basically measure the DERIVATIVE of the visualization function (how much the vis changes when the data changes that much)
% -in both cases, we found that the 2 qual criteria are multually competing; also, we did benchmarks, and found there’s no clear winner in the existing algos; some optimize vis qual but thrash stability, some the other way round, some are kind of in the middle
% -in both cases, we proposed improvements to existing algos which overall increased BOTH qual metrics (vis qual, stability)

% Then, of course, you can add something in this section concerning

% -limitations: feel free to say what could have been done better 😊
% -future work: feel free to speculate here too 😊
% [I can of course suggest a lot here, but I don’t want to monopolize the writing]

% Anyways, the conclusion should be like 3-4 pages, not more.

In this thesis, we considered the visualization of two types of dynamic (time-dependent) data present in information visualization -- weighted hierarchies and high-dimensional datasets. Both data types are ubiquitous in many applications in machine learning, statistics, and data science; and, as identified at the beginning of the thesis, while many techniques have been proposed for the visualization of static (time-independent) forms of both data types, the investigation of techniques that handle the dynamic variants has only been touched in information visualization.

Throughout our work, we found interesting and insightful parallels between the two types of datasets, the challenges they pose to visualization and the visualization evaluation, and also the solutions that we designed to handle both.

Chapter~\ref{ch:soft-eval} kickstarted our work by considering treemap algorithms for the visualization of a particular type of data, namely hierarchies mined from evolving software projects. Already in this limited context, we found that quality of a treemapping algorithm contains two components, namely \emph{visual quality}, that captures how well the cells of the treemap are spread over the drawing space to reflect the data values and also generate easily readable patterns; and \emph{stability}, that measures how well the changes in the depicted treemaps follow the changes in the underlying hierarchies. We also found that the two quality aspects are, roughly speaking, in competition with each other: Algorithms that obtain a high visual quality do this by neglecting stability; and very stable algorithms yield a poor visual quality.

For both hierarchies and high-dimensional projections, we found well-established metrics for gauging visual quality in the literature, \emph{i.e.}, when time is not taken into consideration and only the quality of individual layouts is measured. For treemaps, the quality of the layout is well quantified by the aspect ratio of the contained cells, considering that cells closer to squares form a more readable visualization; for projections, visual quality is measured by how well the distances and neighborhoods from the high-dimensional space are preserved by the low-dimensional embedding.

Regarding stability, however, there were no effective methods of quantifying the relationship between \emph{data change} and \emph{visual change}. As such, and recognizing that stability is an as important desirable property for dynamic visualization as their (static) visual quality, we proposed our own stability metrics. Concerning dynamic treemaps, an algorithm is stable if small changes in the input data result in small changes in the layout, that is, data change and layout change correlate positively. Previously proposed stability metrics measured only the layout change and concluded that small layout changes are a sign of a stable algorithm. However, to properly measure stability, we also need to capture the data change and then correlate data and layout change, an endeavor which we approached in Chapters \ref{ch:soft-eval} and \ref{ch:tree-eval}. This exact same principle applies to dynamic projections. However, in this case, we correlate high-dimensional data change to low-dimensional scatterplot layout change (Chapter \ref{ch:proj-eval}).  

As already mentioned, we found out that visual quality and stability are conflicting criteria, both for treemaps and projections. In order to improve stability, both treemapping and dimensionality reduction methods have to sacrifice visual quality, and conversely. Recognizing this challenge, we next aimed to create methods that improve this balance -- that is, yield overall higher stability and visual quality than existing methods in each class. For dynamic hierarchies, we proposed to this end Greedy Insertion Treemaps (Chapter \ref{ch:git}). For multidimensional projections, we proposed the LD-tSNE and PCD-tSNE methods (Chapter \ref{ch:proj-algo}). Greedy Insertion Treemap (or GIT) aims to preserve treemap-cell neighborhoods over time by constructing an initial so-called Layout Tree (LT), a data structure which is incrementally updated as the input tree data changes, so as to minimize undesired treemap-layout changes. Our state-aware GIT method is simple to implement, generic (handles any types of dynamic hierarchies), and fast (compared to the other state-of-the-art methods). For the dynamic projection challenge, both our newly proposed methods leverage the neighborhood-preservation ability of t-SNE for dynamic time-dependent data. LD-tSNE uses guidance in the form of landmarks, and PCD-tSNE uses information given by the Principal Components of the full temporal dataset. Our results show that PCD-tSNE scores a good balance between stability, neighborhood preservation, and distance preservation, making it one of the best suited general methods for dynamic projections, while LD-tSNE allows creating stable and customizable projections via landmarks selection and steering. 

Another common aspect concerning both dynamic hierarchies and dynamic high-dimensional datasets is the difficulty of \emph{evaluation}. This comprises multiple aspects. Besides the availability of suitable quality metrics -- which we solved as described above -- there is also the difficulty of finding good collections of datasets on which to evaluate existing visualization methods. Such so-called benchmarks were introduced -- only very recently -- for static projections. However, no comprehensive benchmarks existed, at the time of writing this thesis, for static treemapping, let alone for dynamic treemapping and dynamic projections. We created and evaluated several such benchmarks, starting with one for dynamic hierarchies obtained from software evolution (Chapter~\ref{ch:soft-eval}), which we extended next to a far more general benchmark for dynamic hierarchies mined from a wide spectrum of application domains (Chapter~\ref{ch:tree-eval}), and ending with a benchmark for dynamic projections (Chapter~\ref{ch:proj-eval}). 
Creating these benchmarks posed both conceptual problems, in terms of how to describe the huge variability of dynamic datasets along a set of independent traits, and next how to sample these traits; but also practical problems, in terms of how to find real-world datasets that sample the universe of these dynamic datasets, and providing actual reference implementations for the tens of algorithms for treemapping and projection that we need to evaluate. While our proposed benchmarks are, definitely, not fully covering the space of possibilities, they are the first in the dynamic treemapping and projection arenas. We made them fully open source (datasets, algorithms, visualization techniques, quality metrics, and obtained results). We argue that these are important resources for the visualization community which can, now, easily compare new and existing algorithms with new and existing datasets for both practical and research-oriented goals.

Lastly, we presented a real-world application our new dynamic projection methods in the context of hyperkinetic movement disorder analysis. These disorders manifest as abnormal involuntary movements that highly affect the quality of life of the people who suffer from them, and computer supported diagnosis is desired given the complexity of their manifestation. 
In Chapter~\ref{ch:nemo}, we described how we transform the data collected during clinical experiments, we proposed a visual analytics tool designed to support the exploration of the this complex dataset, and we showed an example of data exploration that leads to valuable clinical insights.
This is preliminary investigation, and many points are still open e.g., the actual construction of an automatic classifier based on such data, questions regarding the clinical usage of dynamic projections, the design of a sophisticated UIs for medical professionals, among other directions of future work. Nevertheless, our work showed that projections do have the potential to be useful in the exploration of temporal multidimensional data coming from a real-world problem.


\section{Future work}

There are several possible directions for future work:

\noindent\textbf{Streaming data:} For our algorithm designs, we assumed a finite temporal aspect to the time series we are handling, and we allow our algorithms to ``look into the future'' and adapt accordingly to the changes in the data that are yet to come. 
When dealing with streaming data, we don't have this ability, we can't ``look into the future'', and the algorithm must try to adapt to any unpredictable changes in the data, making the design of this class of methods even more challenging.
In addition, studying this class of (underserved) algorithms implies the design of new quality metrics and the collection of new suitable datasets. This applies both to streaming treemaps and streaming projections.

\noindent\textbf{Deep learning dynamic projections:} Recently, we have seen the use of deep neural networks to produce static projections with a significant computational speed-up while maintaining high-quality metrics and out-of-sample capability \citep{MateusEspadoto}. We believe a similar approach could be investigated for dynamic projections, granting similar benefits to the temporal counterpart.

\noindent\textbf{Extending benchmarks:} 
We can extend our benchmarks with new methods, better ways to choose hyperparameters, new datasets, and new metrics. With a larger number of datasets, we can perform robust tests on the impact of dataset traits on the quality of our projections and treemaps. We can also integrate streaming data techniques, streaming datasets, and dedicated task-based tests. 

\noindent\textbf{Improvements to the NEMO data exploration tool:} 
As stated in \cref{ch:nemo}, there are many promising direction for future development that address current limitations in our tool: the current analyses only support \emph{one} selected task and \emph{one} sensor at a time. It would certainly be advantageous to extend our methods to multiple sensors and tasks. This would lead to better understanding of the postural, circumstantial,  and individual aspects of certain involuntary movement occurrences.
Additionally, we focus only on EMG, accelerometer, and 2D video data. Providing support to the extra data modalities extracted during the experiments would certainly be beneficial to phenotypical classification tasks.

