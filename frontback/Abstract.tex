\begingroup

\let\clearpage\relax
\let\cleardoublepage\relax
\let\cleardoublepage\relax

\manualmark
\markboth{\spacedlowsmallcaps{Abstract}}{\spacedlowsmallcaps{Abstract}} 
\phantomsection 

\pdfbookmark[0]{Abstract}{Abstract}
\chapter*{Abstract}

When it comes to tools and techniques designed to help understanding complex abstract data, visualization methods play a prominent role. They enable human operators to leverage their pattern finding, outlier detection, and questioning abilities to visually reason about a given dataset. Many methods exist that create suitable and useful visual representations of \emph{static} abstract, non-spatial, data. However, for \emph{temporal} abstract, non-spatial, datasets, in which the data changes and evolves through time, far fewer visualization techniques exist.

This thesis focuses on the particular cases of temporal hierarchical data representation via dynamic treemaps, and temporal high-dimensional data visualization via dynamic projections. We tackle the joint question of how to extend projections and treemaps to stably, accurately, and scalably handle temporal multivariate and hierarchical data. In both cases, the literature for the static techniques is rich and the state-of-the-art methods have proven to be valuable tools in data analysis. Their temporal/dynamic counterparts, however, are not as well studied, and, until recently, there were few hierarchical and high-dimensional methods that explicitly took into consideration the temporal aspect of the data. In addition, there are few or no metrics to assess the quality of these temporal mappings, and even fewer comprehensive benchmarks to compare these methods. 

This thesis addresses the abovementioned shortcomings. For both dynamic treemaps and dynamic projections. We propose ways to accurately measure temporal stability; we evaluate existing methods considering the tradeoff between stability and visual quality; and we propose new methods that strike a better balance between stability and visual quality than existing state-of-the-art techniques. We demonstrate our methods with a wide range of real-world data, including an application of our new dynamic projection methods to support the analysis and classification of hyperkinetic movement disorder data.

\newpage

\selectlanguage{dutch}

\manualmark
\markboth{\spacedlowsmallcaps{Samenvatting}}{\spacedlowsmallcaps{Samenvatting}} 
\phantomsection 

\pdfbookmark[0]{Samenvatting}{samenvatting}
\chapter*{Samenvatting}

Thesis abstract in Dutch.

\newpage

\selectlanguage{portuguese}

\manualmark
\markboth{\spacedlowsmallcaps{Resumo}}{\spacedlowsmallcaps{Resumo}} 
\phantomsection 

\pdfbookmark[0]{Resumo}{resumo}
\chapter*{Resumo}

Quando se trata de ferramentas e técnicas projetadas para ajudar na compreensão dados abstratos complexos, métodos de visualização desempenham um papel proeminente. Eles permitem que os operadores humanos alavanquem suas habilidades de descoberta de padrões, detecção de valores discrepantes, e questionamento visual para a raciocinar sobre um determinado conjunto de dados. 
Existem muitos métodos que criam representações visuais adequadas e úteis de para dados \emph{estáticos}, abstratos, e não-espaciais. No entanto, para dados \emph{temporais}, abstratos, e não-espaciais, isto é, dados que mudam e evoluem no tempo, existem poucas técnicas apropriadas.

Esta tese concentra-se nos casos específicos de representação temporal de dados hierárquicos por meio de treemaps dinâmicos, e visualização temporal de dados de alta dimensionalidade via projeções dinâmicas. Nós abordar a questão conjunta de como estender projeções e treemaps de forma estável, precisa e escalável para lidar  com conjuntos de dados hierárquico-temporais e multivariado-temporais.
Em ambos os casos, a literatura para técnicas estáticas é rica e os métodos estado da arte provam ser ferramentas valiosas em análise de dados. Suas contrapartes temporais/dinâmicas, no entanto, não são tão bem estudadas e, até recentemente, existiam poucos métodos hierárquicos e de alta dimensão que explicitamente levavam em consideração o aspecto temporal dos dados. Além disso, existiam poucas métricas para avaliar a qualidade desses mapeamentos visuais temporais, e ainda menos benchmarks abrangentes para comparação esses métodos. 

Esta tese aborda as deficiências acima mencionadas para treemaps dinâmicos e projeções dinâmicas. Propomos maneiras de medir com precisão a estabilidade temporal; avaliamos os métodos existentes, considerando o compromisso entre estabilidade e qualidade visual; e propomos novos métodos que atinjem um melhor equilíbrio entre estabilidade e a qualidade visual do que as técnicas estado da arte atuais. Demonstramos nossos métodos com uma ampla gama de dados do mundo real, incluindo uma aplicação de nossos novos métodos de projeção dinâmica para apoiar a análise e classificação dos dados de transtorno de movimentos. 


\selectlanguage{english}

\endgroup			
