\begingroup

\let\clearpage\relax
\let\cleardoublepage\relax
\let\cleardoublepage\relax

\manualmark
\markboth{\spacedlowsmallcaps{Abstract}}{\spacedlowsmallcaps{Abstract}} 
\phantomsection 

\pdfbookmark[0]{Abstract}{Abstract}
\chapter*{Abstract}


When it comes to tools and techniques designed to help us undestand complex abstract data, visualization methods play a prominent role, allowing us, as human operators, to leverage on our pattern finding, outlier detection, and questioning abilities to visually reason about a given dataset.
Many visualization methods exist that create suitable and useful visual representation of \emph{static} abstract data. However, for \emph{temporal} datasets, that is, when the data changes and evolves through time, these unadapted visual mappings tend to create unstable and untruthful representations, harmful to the understanding of the data structure and evolution.

This thesis focus on the particular cases of temporal hierarchical data representation via dynamic treemaps, and temporal high-dimensional data visualization via dynamic projections, as we tackle the question of how to extend projections and treemaps to stably, accurately, and scalably handle temporal multivariate and hierarchical data.
In both cases, the literature for the static techniques is rich and the state-of-the-art methods have proven to be valuable tools in data analysis. Their temporal/dynamic conterparts, however, are not as well studied, and, until recently, there were few hierarchical and high-dimensional methods that explicitly took into consideration the temporal aspect of the data. In addition, there were few or no metrics to assess the quality of these temporal mappings, and there were no comprehensible benchmarks to compare these methods. This thesis addresses all these shortcomings.

On both dynamic treemap and dynamic projection fronts, we propose ways of accurately measuring temporal stability, we evaluate the methods in the literature considering the tradeoff between stability and visual quality, and we design state-of-the-art methods that strike a good balance between stability and visual quality.

Last but not least, we present a real world application of our new dynamic projection methods as we develop a tool to support analysis and classification of hyperkinetic movement disorder data.


\newpage

\selectlanguage{dutch}

\manualmark
\markboth{\spacedlowsmallcaps{Samenvatting}}{\spacedlowsmallcaps{Samenvatting}} 
\phantomsection 

\pdfbookmark[0]{Samenvatting}{samenvatting}
\chapter*{Samenvatting}

Thesis abstract in Dutch.

\newpage

\selectlanguage{portuguese}

\manualmark
\markboth{\spacedlowsmallcaps{Resumo}}{\spacedlowsmallcaps{Resumo}} 
\phantomsection 

\pdfbookmark[0]{Resumo}{resumo}
\chapter*{Resumo}

Thesis abstract in Portuguese.

\selectlanguage{english}

\endgroup			
